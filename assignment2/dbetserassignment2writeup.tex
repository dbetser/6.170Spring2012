\documentclass[11pt,letterpaper]{article}
\usepackage[latin1]{inputenc}
\usepackage{amsmath}
\usepackage{amsfonts}
\usepackage{amssymb}
\usepackage{graphicx}
\usepackage{capt-of}

\setlength{\parskip}{1pc}
\setlength{\parindent}{0pt}
\setlength{\topmargin}{-3pc}
\setlength{\textheight}{9.0in}
\setlength{\oddsidemargin}{0pc}
\setlength{\evensidemargin}{0pc}
\setlength{\textwidth}{6.5in}

\title{6.170 Assignment 2: Object Models}
\author{Dina Betser}


\begin{document}
\maketitle

\section{Background}
\begin{enumerate}
\item Entity Relationship Model\\
ER modeling is used to describe the type of information that is to be stored in a database. Objects are represented as ``entities''.
\item Object Modeling Technique\\
This was developed as a method to develop object-oriented systems and to support OOP. Objects and their relationships are expressed using multiplicities.
\item Software Analysis Patterns\\
This modeling technique aims to represent ideas that have been useful in one practical context and will probably be useful in others. Objects
\item Unified Modeling Language\\

\end{enumerate}

One feature that is present in all modeling formats except for ER modeling is inheritance/generalization.

\section{Conceptual modeling problems}
\begin{itemize}
\item Prerequisites. Model the relationships between classes and their prerequisites. Note that there may be different ways in which the prerequisites of a class can be satisfied; for example, two prerequisite classes may be interchangeable.\\

\item Street map. Model a street map that includes named streets and their intersections, and the notion of one-way streets and divided highways. Note in particular that one street may be accessible from an intersecting street only for traffic moving in a particular direction.
\item Voting ballots. Model the ballots cast in a voting scheme, in which on each ballot the voter makes choices of candidates for a variety of offices.
\item Java types. Model the type structure of objects in a Java program, in which there are two kinds of types, class types and interfaces, related by implements and extends. Include in your model a notion of variables, each with a declared type and containing an object of a given type. What is the relationship between the two types?\\

\end{itemize}



\section{Modeling Python Modules}


\section{Extracting an OM from an API}


\section{Metamodeling}

\end{document}

