\documentclass[11pt,letterpaper]{article}
\usepackage[latin1]{inputenc}
\usepackage{amsmath}
\usepackage{amsfonts}
\usepackage{amssymb}
\usepackage{graphicx}
\usepackage{capt-of}

\setlength{\parskip}{1pc}
\setlength{\parindent}{0pt}
\setlength{\topmargin}{-3pc}
\setlength{\textheight}{9.0in}
\setlength{\oddsidemargin}{0pc}
\setlength{\evensidemargin}{0pc}
\setlength{\textwidth}{6.5in}

\title{6.170 Assignment 1 Documentation}
\author{Dina Betser}


\begin{document}
\maketitle

\section{Models}
\subsection{Object Models}
The object model for the problem domain is included in Figure \ref{fig:ob1}. The problem domain object model demonstrates the system that must be built. A photo \texttt{Gallery} corresponding to a \texttt{Directory} of \texttt{Photo}s. Each \texttt{Photo} in the directory is preceded by the previous \texttt{Photo} in the \texttt{Directory} and followed by the next \texttt{Photo}. Every \texttt{Photo} has a \texttt{Caption} made up of metadata fields from the set of \texttt{IPTC Info Elements}\footnote{The IPTC Info elements include: date created, digital creation date, reference number, custom8, custom9, sub-location, object cycle, custom4, custom5, custom6, custom7, custom1, custom2, reference date, by-line title, local caption, keywords, province/state, category, custom17, custom14, digital creation time, custom12, custom13, custom10, custom11, headline, custom18, custom19, source, contact, by-line, object name, content location code, language identifier, release date, expiration date, reference service, custom16, original transmission reference, originating program, subject reference, city, supplemental category, content location name, country/primary location code, editorial update, custom15, fixture identifier, custom3, country/primary location name, action advised, custom20, copyright notice, program version, image orientation, edit status, expiration time, release time, credit, time created, special instructions, writer/editor, caption/abstract, urgency, and image type.}.
\begin{center}
\includegraphics[width=150pt]{dot/obmodproblem.png}
\label{fig:ob1} 
\end{center}

I was not sure what the abstract state object module should look like, as the final application is composed of static HTML, which does not store state.

\section{Design Notes}
\subsection{Key Challenges}
\begin{itemize}
\item Getting acquainted with Jinja and valid HTML best practices. 
\item Organizing the code to encapsulate the interaction with the file system, the Jinja templates, and the image data.
\item Using just HTML/CSS to show one image at a time. Reused a CSS code snippet for this.
\item Representing an image with all of its necessary components.
\end{itemize}

\subsection{Issues Arising}
\begin{itemize}
\item Images that did not have values for the desired metadata fields.\\
For images without values for desired metadata fields, I simply returned a default message stating that the field was empty.
\item HTML did not validate using the validator.\\
Previously, I was using the doctype declaration of \texttt{<!DOCTYPE html>}. I switched the doctype to \texttt{<!DOCTYPE html PUBLIC "-//W3C//DTD HTML 4.01 Transitional//EN">} to resolve the validation issues. I also needed to do things like ensure that the title of any image did not start with a number, so I prepended the word ``photo'' to ensure that all titles started with a valid character.
\item Storing a pointer to the previous and next photo in the directory.\\
In the first version of my implementation, I did not facilitate ImageDataObjects being aware of the predecessors and successors within the directory. I added a \texttt{prev} and \texttt{next} field in the \texttt{ImageDataObject} class to solve this issue.
\end{itemize}

\subsection{Critique}
\begin{itemize}
\item This project required a clear understanding of each Python class that was used. Each class I created had a clear specification and purpose, so the code itself was organized fairly well.

The implementation was the most basic of the 
\end{itemize}
\section{Specification}
\subsection{Overview}
\subsection{Key Features}
The key features of this implementation of the 6.170 photo gallery are:
\begin{itemize}
\item Minimalist, intuitive scrolling interface.
\item Images are 
\item Generated HTML depends on multiple command-line arguments that can customize the files used and
\end{itemize}
\subsection{User Manual}
To run the code, python2.6 must be installed, with the jinja2 module and the iptcinfo module available.

The 6.170 photo gallery is a static HTML page that is generated via a Python script that uses the Jinja templating system. The HTML is generated by running a python script with command-line arguments specifying options.
For example, a sample run of the script to generate the html file photo\_gallery.html might look like:
\begin{verbatim}
$ python2.6 photo_gallery.py images/ photo_gallery.html caption/abstract templates/photo_gallery_template.html
\end{verbatim}
In this case, the arguments are as follows:
\begin{verbatim}
$ python2.6 photo_gallery.py <image_directory> <output_filename> <comma-separated list of metadata fields> <input_filename>
\end{verbatim}

The user may only specify a single directory from which to generate the gallery, and no subdirectories are included in the gallery. This image directory must be specified with a slash (\/) following the name.


\section{Implementation}

\subsection{Module Dependency Diagram}
This code's modules can only be seen as the member variables of classes, since there was only one Python module involved in the code generation.
\begin{center}
\includegraphics[width=150pt]{dot/moddepdiagram.png}
\label{fig:ob2} 
\end{center}
\subsection{Code Notes}
This code was fairly straightforward to write. The biggest issue I had was determining how to display only one photo, with previous and next buttons, using only HTML and CSS.

\begin{itemize}
\item 
\end{itemize}



\section{Testing}

\subsection{Test Plan}
Test the generated HTML in multiple browsers.
\subsection{Test Cases}
\subsection{Rationale and Conclusions}

\end{document}

